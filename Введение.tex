\section*{ВВЕДЕНИЕ}
\addcontentsline{toc}{section}{ВВЕДЕНИЕ}

Современное общество невозможно представить без информационных технологий, которые активно проникают во все сферы жизни и деятельности человека. Одной из важнейших задач в рамках информатизации является эффективное хранение, обработка и представление данных. Особую роль в этом процессе играют системы управления базами данных (СУБД), позволяющие упорядоченно работать с большими объемами структурированной информации.

На протяжении десятилетий развивались как промышленные, так и прикладные решения в области СУБД, от мощных серверных систем до встраиваемых решений. Однако не всегда требуется развертывание сложной инфраструктуры с использованием сторонних серверов, реляционных систем и облачных технологий. Во многих случаях достаточно легковесной, локальной и интуитивно понятной системы, способной выполнять базовые операции с таблицами и предоставлять визуальный интерфейс взаимодействия с данными.

Актуальность данной темы обусловлена необходимостью создания простой и надежной программной среды для управления локальными базами данных, ориентированной на широкую аудиторию пользователей — от студентов и преподавателей до специалистов, решающих локальные задачи автоматизации.

\emph{Цель настоящей работы} – разработка простой СУБД с графическим интерфейсом, способной обеспечивать базовые операции с таблицами: вставку, выборку, обновление и удаление данных. Для достижения поставленной цели необходимо решить \emph{следующие задачи:}
\begin{itemize}
\item провести анализ предметной области;
\item определить структуру хранимых данных и поддерживаемые типы данных;
\item обеспечить поддержку основных операций над таблицами: вставки, выборки, обновления и удаления данных по условиям;
\item спроектировать настольное приложение для СУБД;
\item реализовать приложение, используя графический интерфейс.
\end{itemize}

\emph{Структура и объем работы.} Отчет состоит из введения, 4 разделов основной части, заключения, списка использованных источников, 2 приложений. Текст выпускной квалификационной работы равен \formbytotal{lastpage}{страниц}{е}{ам}{ам}.

\emph{Во введении} сформулирована цель работы, поставлены задачи разработки, описана структура работы, приведено краткое содержание каждого из разделов.

\emph{В первом разделе} на стадии анализа предметной области рассматриваются различные виды СУБД, области их применения, история и перспективы их развития.

\emph{Во втором разделе} на стадии технического задания приводятся требования к разрабатываемой системе.

\emph{В третьем разделе} на стадии технического проектирования представлены проектные решения для СУБД.

\emph{В четвертом разделе} приводится список классов и их методов, использованных при разработке СУБД, производится тестирование разработанного настольного приложения.

В заключении излагаются основные результаты работы, полученные в ходе разработки.

В приложении А представлен графический материал.
В приложении Б представлены фрагменты исходного кода. 
