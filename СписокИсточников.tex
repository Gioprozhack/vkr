\addcontentsline{toc}{section}{СПИСОК ИСПОЛЬЗОВАННЫХ ИСТОЧНИКОВ}

\begin{thebibliography}{9}

    \bibitem{} Кореньков В. В., Иванцова О. В., Филозова И. А. Технологии баз данных. Проектирование реляционных баз данных. – 2022. – Текст~: непосредственный.
    \bibitem{} Осипов Д. Технологии проектирования баз данных. – 2022. – Текст~: непосредственный.
    \bibitem{} Илюшечкин В. Основы использования и проектирования баз данных. Учебник для академического бакалавриата. – 2021. – Текст~: непосредственный.
    \bibitem{} Жидченко Т. В. Базы данных. – 2021. – Текст~: непосредственный.
	\bibitem{} Стасышин В. М. Разработка информационных систем и баз данных. – 2020. – Текст~: непосредственный.
	\bibitem{} Волхонский А. Н. Модели данных при проектировании баз данных автоматизированных систем //Международный студенческий научный вестник. – 2021. – №. 6. – С. 38. – Текст~: непосредственный.
	\bibitem{} Годин В. В., Стружкин Н. П. Базы данных: проектирование. – 2019. – Текст~: непосредственный.
	\bibitem{} Лазицкас Е. А., Загумённикова И. Н., Гилевский П. Г. Базы данных и системы управления базами данных. – 2016. – Текст~: непосредственный.
	\bibitem{} Махкамов Ш. Теоретические основы базы данных (мб) и системы управления базами данных (мббт) //Информатика и инженерные технологии. – 2023. – Т. 1. – №. 1. – С. 90-94. – Текст~: непосредственный.    
	\bibitem{} Тораев М. Ш. Основы реляционных баз данных //КОНЦЕПЦИИ УСТОЙЧИВОГО РАЗВИТИЯ НАУКИ В СОВРЕМЕННЫХ УСЛОВИЯХ. – 2018. – С. 208-210. – Текст~: непосредственный.    
	\bibitem{} Бизли Д. Python. Исчерпывающее руководство. – Питер, 2023. - Текст~: непосредственный.
	\bibitem{} МакГрат М. Python. Программирование для начинающих. – Litres, 2015. - Текст~: непосредственный.
	\bibitem{} Рамальо Л. Python. К вершинам мастерства. – Litres, 2022. - Текст~: непосредственный.
	\bibitem{} Персиваль Г. Python. Разработка на основе тестирования. – Litres, 2022. - Текст~: непосредственный.
	\bibitem{} Берман К. Основы Python для Data Science. – Питер, 2023. - Текст~: непосредственный.
	\bibitem{} Бадд Т. Объектно-ориентированное программирование //электронная книга. – 1997. - Текст~: непосредственный.
	\bibitem{} Бежанова Е. Х. Объектно–ориентированное программирование. – 2001. - Текст~: непосредственный.
	\bibitem{} Задорожный С. С., Фадеев Е. П. Объектно-ориентированное программирование на языке Python //М.: Физический факультет МГУ им. МВ Ломоносова. – 2022. - Текст~: непосредственный.
	\bibitem{} КАРЧАГИН В. С., ТУРУШЕВ Т. К., БОГДАНОВА В. С. ОСОБЕННОСТИ ОБЪЕКТНО-ОРИЕНТИРОВАННОГО ПРОГРАММИРОВАНИЯ В PYTHON //Приоритеты современной науки: актуальные исследования и направления. – 2021. – С. 58-60. - Текст~: непосредственный.
	\bibitem{} Костырева С. А., Курьян И. С., Негина Д. В. Объектно-ориентированное программирование в Python //ТЕОРИЯ И ПРАКТИКА СОВРЕМЕННОЙ НАУКИ. – 2022. – С. 52-54. - Текст~: непосредственный.
	
\end{thebibliography}
