\abstract{РЕФЕРАТ}

Объем работы равен \formbytotal{lastpage}{страниц}{е}{ам}{ам}. Работа содержит \formbytotal{figurecnt}{иллюстраци}{ю}{и}{й}, \formbytotal{tablecnt}{таблиц}{у}{ы}{}, \arabic{bibcount} библиографических источников и \formbytotal{числоПлакатов}{лист}{}{а}{ов} графического материала. Количество приложений – 2. Графический материал представлен в приложении А. Фрагменты исходного кода представлены в приложении Б.

Перечень ключевых слов: база данных, СУБД, интерфейс, таблица, сериализация, Python, команды, SELECT, INSERT, DELETE, структура данных, сохранение, обработка,выражение, логика, объект, файл, .db, строка, тип данных.

Объектом разработки является программная система — настольное приложение, представляющее собой простую в использовании систему управления базами данных, предназначенную для хранения, обработки и визуализации табличных данных.

Целью выпускной квалификационной работы является разработка простой СУБД с графическим интерфейсом, реализованной средствами языка Python, способной обеспечивать базовые операции с базой данных: вставку, выборку, обновление и удаление данных.

В процессе создания системы была разработана архитектура базы данных и приложение с графическим интерфейсом для взаимодействия пользователя с системой.

\selectlanguage{english}
\abstract{ABSTRACT}
  
The volume of work is \formbytotal{lastpage}{page}{}{s}{s}. The work contains \formbytotal{figurecnt}{illustration}{}{s}{s}, \formbytotal{tablecnt}{table}{}{s}{s}, \arabic{bibcount} bibliographic sources and \formbytotal{числоПлакатов}{sheet}{}{s}{s} of graphic material. The number of applications is 2. The graphic material is presented in annex A. The layout of the site, including the connection of components, is presented in annex B.

List of keywords: database, DBMS, interface, table, serialization, Python, commands, SELECT, INSERT, DELETE, data structure, saving, processing, expression, logic, object, file, .db, string, data type.

The object of the development is a software system — a desktop application representing a user-friendly database management system designed for storing, processing, and visualizing tabular data.

The goal of this work is to develop a simple DBMS with a graphical user interface implemented in Python, capable of performing basic table operations: inserting, selecting, updating, and deleting data, as well as saving the database to disk and loading it later.

During the development process, the architecture of the database was designed, along with a graphical interface application for user interaction with the system.

\selectlanguage{russian}
