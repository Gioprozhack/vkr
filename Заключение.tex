\section*{ЗАКЛЮЧЕНИЕ}
\addcontentsline{toc}{section}{ЗАКЛЮЧЕНИЕ}

Развитие информационных технологий и рост объемов данных способствовали повышенному интересу к созданию и совершенствованию систем управления базами данных. Базы данных стали неотъемлемой частью практически всех сфер человеческой деятельности — от бизнеса и государственного управления до научных исследований и повседневной жизни. Умение организовывать, хранить и обрабатывать информацию в структурированной форме имеет решающее значение для повышения эффективности процессов и принятия обоснованных решений.

В условиях возрастающих требований к гибкости, простоте и автономности обработки данных разработка собственных, легковесных и адаптируемых СУБД приобретает всё большую актуальность.

В рамках данной выпускной квалификационной работы была разработана собственная система управления базами данных реляционной модели, реализующая базовые функции создания, хранения, изменения и выборки данных.

Основные результаты работы:

\begin{enumerate}
\item Проведен анализ предметной области. Проведено исследование причин разработки баз данных, областей их применения и перспектив их развития.
\item Разработана концептуальная модель программной системы, определены основные требования к системе и структуре базы данных.
\item Осуществлено проектирование программной системы. Разработана архитектура настольного приложения и базы данных. Разработан пользовательский интерфейс приложения.
\item Реализована программная система, проведено модульное и системное тестирование СУБД.
\end{enumerate}

Все требования, объявленные в техническом задании, были полностью реализованы. Все задачи, поставленные в начале разработки проекта, были решены.

Готовый рабочий проект представлен в виде настольного приложения с графическим интерфейсом.
