\section{Рабочий проект}

\subsection{Спецификация компонентов и классов программы}

\subsubsection{Модуль main.py}

Модуль предоставляет графический интерфейс для управления базами данных. 

Класс модуля -- MainWindow.

Описание класса MainWindow.
Класс предназначен для управления главным окном приложения и передачи пользовательских команд в обработку. Базовый класс -- Tk, стандартный класс библиотеки Tkinter. Интерфейсы: панель меню, позволяющая выбрать файл базы данных для работы и запустить обработку команды; виджет Treeview, отображающий данные выбранной таблицы; виджет Text, куда пользователь вводит команды. Константы отсутствуют. Внутренние поля представлены в таблице~\ref{table:main_widgets}.

\begin{xltabular}{\textwidth}{|X|X|X|}
	\caption{Внутренние поля класса MainWindow\label{table:main_widgets}} \\
	\hline 
	\centrow Внутреннее поле & 
	\centrow Тип & 
	\centrow Описание \\ 
	\hline 
	\endfirsthead
	
	\caption*{Продолжение таблицы \ref{table:main_widgets}} \\
	\hline 
	\centrow Внутреннее поле & 
	\centrow Тип & 
	\centrow Описание \\ 
	\hline 
	\endhead
	
	selected\_db & Database & Активная база данных, над которой проводятся операции \\ \hline
	main\_menu & Menu & Меню, содержащее пункт для выбора базы данных и пункт для запуска команды \\ \hline
	table\_frame & Frame & Фрейм, содержащий в себе табличный виджет Treeview и скроллбары для его прокрутки \\ \hline
	table & Treeview & Виджет для визуального отображения таблиц и выборки данных \\ \hline
	ysb & Scrollbar & Скроллбар для вертикальной прокрутки таблицы \\ \hline
	xsb & Scrollbar & Скроллбар для горизонтальной прокрутки таблицы \\ \hline
	sql\_field\_frame & Frame & Фрейм, содержащий текстовое поле для ввода команд \\ \hline
	sql\_field & Text & Текстовое поле для ввода команд \\ \hline
\end{xltabular}
Методы класса представлены в таблице~\ref{table:main_method}.
\renewcommand{\arraystretch}{0.8} % уменьшение расстояний до сетки таблицы
\begin{xltabular}{\textwidth}{|>{\hsize=0.85\hsize\raggedright\arraybackslash}X|
		>{\hsize=0.85\hsize\setlength{\baselineskip}{0.7\baselineskip}}X|
		>{\hsize=1.0\hsize}X|
		>{\hsize=1.3\hsize}X|}
	\caption{Методы класса MainWindow\label{table:main_method}}\\
	\hline 
	\centrow \setlength{\baselineskip}{0.7\baselineskip} Название метода & 
	\centrow Параметры метода &
	\centrow Возвращаемое значение & 
	\centrow Назначение метода \\ 
	\hline 
	\endfirsthead
	
	\caption*{Продолжение таблицы \ref{table:main_method}}\\
	\hline 
	\centrow Название метода & 
	\centrow Параметры метода &
	\centrow Возвращаемое значение & 
	\centrow Назначение метода \\ 
	\hline 
	\endhead
	
	\_\_init\_\_ & Не имеет & Не имеет  & Инициализирует графический интерфейс приложения, включая его макет и элементы \\ \hline 
	update\_table & headings -- названия столбцов отображаемой таблицы; contents -- содержимое отображаемой таблицы & Не имеет & Обновляет визуальное отображение таблицы в интерфейсе \\ \hline
	select\_database & Не имеет & Не имеет & Вызывает окно для выбора файла базы данных \\ \hline
	execute\_sql & Не имеет & Не имеет & Вызывает функцию обработки команд из модуля parser.py \\ \hline
	
\end{xltabular}
\renewcommand{\arraystretch}{1.0} % восстановление сетки
\vspace{-\baselineskip}

\subsubsection{Модуль parser.py}

Модуль отвечает за обработку пользовательских команд. Не содержит классов и констант. Методы модуля представлены в таблице~\ref{table:parser_method}.
\begin{xltabular}{\textwidth}{|>{\hsize=0.85\hsize\raggedright\arraybackslash}X|
		>{\hsize=0.85\hsize\setlength{\baselineskip}{0.7\baselineskip}}X|
		>{\hsize=1.0\hsize}X|
		>{\hsize=1.3\hsize}X|}
	\caption{Методы модуля parser.py\label{table:parser_method}}\\
	\hline 
	\centrow \setlength{\baselineskip}{0.7\baselineskip} Название метода & 
	\centrow Параметры метода &
	\centrow Возвращаемое значение & 
	\centrow Назначение метода \\ 
	\hline 
	\endfirsthead
	
	\caption*{Продолжение таблицы \ref{table:parser_method}}\\
	\hline 
	\centrow Название метода & 
	\centrow Параметры метода &
	\centrow Возвращаемое значение & 
	\centrow Назначение метода \\ 
	\hline 
	\endhead
	
	eval\_node & node -- условие where; row (значение по умолчанию -- None) -- строка таблицы, из которой нужно взять значения по столбцам & Результат логического выражения в условии where  & Вычисляет значение логического выражения в условии where для переданной строки таблицы \\ \hline 
	parse\_command & command -- команда, введённая пользователем в текстовое поле; db -- движок активной на данный момент базы данных, которому будут передаваться инструкции & Названия столбцов и содержимое для таблицы в интерфейсе & Обрабатывает пользовательские команды и передаёт инструкции для выполнения в движок базы данных \\ \hline	
\end{xltabular}
\renewcommand{\arraystretch}{1.0} % восстановление сетки
\vspace{-\baselineskip}

\subsubsection{Модуль database.py}

Модуль представляет собой движок, реализующий операции над файлом базы данных.

Класс модуля: Database.

Константы модуля представлены в таблице~\ref{table:db_const}.

\renewcommand{\arraystretch}{0.8}
\begin{xltabular}{\textwidth}{|>{\hsize=1.1\hsize\raggedright\arraybackslash}X|>{\hsize=0.95\hsize\raggedright\arraybackslash}X|>{\hsize=0.95\hsize\raggedright\arraybackslash}X|}
	\caption{Константы модуля database.py\label{table:db_const}} \\
	\hline 
	\centrow Имя константы & 
	\centrow Тип & 
	\centrow Описание \\ 
	\hline 
	\endfirsthead
	
	\caption*{Продолжение таблицы \ref{table:db_const}} \\
	\hline 
	\centrow Имя константы & 
	\centrow Тип & 
	\centrow Описание \\ 
	\hline 
	\endhead
	
	TABLE\_META\_SIZE & int & Размер метаданных таблицы в байтах \\ \hline
	MAX\_TABLE\_COUNT & int & Максимально доступное количество таблиц в базе данных \\ \hline
	MAX\_COLUMN\_COUNT & int & Максимально доступное количество столбцов в таблице \\ \hline
	PAGE\_SIZE & int & Размер одной страницы таблицы в байтах \\ \hline
	DATA\_TYPES & dict & Словарь для сопоставления названий типов данных с их индексом в файле БД \\ \hline
	STRUCT\_TYPES & dict & Словарь для сопоставления типов данных с форматом упаковки значений соответствующего типа данных в бинарный формат \\ \hline
	CHECK\_TYPES & dict & Словарь для проверки корректности типов данных передаваемых для записи в таблицу значений \\ \hline
	DEAD\_END & int & Значение, которое указывается в последней странице таблицы \\ \hline
\end{xltabular}
\renewcommand{\arraystretch}{1.0} % восстановление сетки
\vspace{-\baselineskip}

Описание класса Database. Класс предназначен для выполнения операций над файлом базы данных. Не наследуется от других классов. Интерфейсы -- общедоступные методы \_\_init\_\_, create\_table, insert, select, update, delete, drop\_table. Константы отсутствуют. Внутреннее поле класса -- filepath (тип str), оно хранит путь к файлу базы данных. Методы класса представлены в таблице~\ref{table:db_method}.

\renewcommand{\arraystretch}{0.8} % уменьшение расстояний до сетки таблицы
\begin{xltabular}{\textwidth}{|>{\hsize=0.85\hsize\raggedright\arraybackslash}X|
		>{\hsize=0.85\hsize\setlength{\baselineskip}{0.7\baselineskip}}X|
		>{\hsize=1.0\hsize}X|
		>{\hsize=1.3\hsize}X|}
	\caption{Методы класса Database\label{table:db_method}}\\
	\hline 
	\centrow \setlength{\baselineskip}{0.7\baselineskip} Название метода & 
	\centrow Параметры метода &
	\centrow Возвращаемое значение & 
	\centrow Назначение метода \\ 
	\hline 
	\endfirsthead
	
	\caption*{Продолжение таблицы \ref{table:db_method}}\\
	\hline 
	\centrow Название метода & 
	\centrow Параметры метода &
	\centrow Возвращаемое значение & 
	\centrow Назначение метода \\ 
	\hline 
	\endhead
	
	\_\_init\_\_ & path -- путь к файлу БД & Не имеет  & Инициализирует движок для работы с файлом БД по указанному пути, если такого файла нет, то создаёт его \\ \hline 
	\_allocate\_page & Не имеет & Не имеет & Добавляет пустую страницу в файл БД \\ \hline
	create\_table & table\_name -- имя таблицы; columns -- названия столбцов и их типы данных & Не имеет & Создаёт новую таблицу с указанными именем и столбцами \\ \hline
	insert & table\_name -- имя таблицы; values -- значения, которые будут вставлены в таблицу & Не имеет & Добавляет в страницу новую запись с указанными значениями \\ \hline
	select & table\_name -- имя таблицы; columns -- список столбцов для выборки; where (значение по умолчанию -- lambda row: True) -- условие, по которому будет производиться выборка & Названия и типы данных выбранных столбцов; данные, удовлетворяющие условию & Производит выборку данных в указанных столбцах из строк, удовлетворяющих условию \\ \hline
	update & table\_name -- имя таблицы; updated\_values -- новые значения; where (значение по умолчанию -- lambda row: True) -- условие, по которому будет производиться обновление & Не имеет & Обновляет данные в указанных столбцах в строках, удовлетворяющих условию \\ \hline	
	delete & table\_name -- имя таблицы; where (значение по умолчанию -- lambda row: True) -- условие, по которому будет производиться удаление & Не имеет & Удаляет из таблицы все строки, удовлетворяющие условию \\ \hline
	drop\_table & table\_name -- имя таблицы & Не имеет & Удаляет указанную таблицу из базы данных \\ \hline
	
\end{xltabular}
\renewcommand{\arraystretch}{1.0} % восстановление сетки
\vspace{-\baselineskip}

\subsection{Модульное тестирование разработанной программной системы}

Модульный тест для класса Database из модуля database.py представлен в таблице~\ref{table:db_test}.

\renewcommand{\arraystretch}{0.8}
\begin{xltabular}{\textwidth}{|X|X|X|X|}
	\caption{Модульное тестирование класса Database\label{table:db_test}} \\
	\hline
	\centrow Описание теста &
	\centrow Тестируемая функция & 
	\centrow Входные данные & 
	\centrow Ожидаемый результат \\ 
	\hline 
	\endfirsthead
	
	\caption*{Продолжение таблицы \ref{table:db_test}} \\
	\hline 
	\centrow Описание теста & 
	\centrow Тестируемая функция & 
	\centrow Входные данные & 
	\centrow Ожидаемый результат \\ 
	\hline 
	\endhead
	
	Проверка инициализации базы данных & \_\_init\_\_ & Путь к новому файлу &  Файл создан, количество таблиц в созданной БД: 0, индекс свободной страницы: 0 \\ \hline
	Проверка создания таблицы & create\_table & Название и словарь из 3 колонок & Таблица создана, метаданные корректны \\ \hline
	Проверка обработки ошибок при создании таблицы & create\_table & 1. Слишком длинное имя таблицы\newline2. Слишком много колонок\newline3. Слишком длинное имя колонки\newline4. Неизвестный тип данных & Соответствующая ошибка в каждом случае \\ \hline
	Проверка вставки и выборки данных & insert, select & 3 записи в таблицу & Корректное количество и содержимое записей \\ \hline
	Проверка выборки с указанием колонок & select & Названия нужных колонок & Данные только из указанных колонок \\ \hline
	Проверка выборки с фильтром (where) & select & Условие where для выборки & Все записи, соответствующие условию \\ \hline
	Проверка обработки ошибок при вставке & insert & 1. Несуществующая таблица\newline2. Неверное количество значений\newline3. Несоответствие типов & Соответствующая ошибка в каждом случае \\ \hline
	Проверка обновления записей по условию & update & Данные для замены и условие where & Все записи, соответствующие условию, обновлены новыми данными \\ \hline
	Проверка обработки ошибок при обновлении & update & 1. Несуществующая таблица\newline2. Несуществующая колонка\newline3. Несоответствие типов & Соответствующая ошибка в каждом случае \\ \hline
	Проверка удаления данных по условию & delete & Таблица и условие where & Данные удалены в соответствии с условиями \\ \hline
	Проверка удаления таблицы & drop\_table & Таблица с 1 записью & Таблица удалена, обращение к ней вызывает ошибку, количество таблиц в файле: 0 \\ \hline
	Проверка работы с несколькими таблицами & create\_table, insert, select & 2 таблицы с разными данными & Данные в таблицах не пересекаются, операции работают независимо \\ \hline
\end{xltabular}
\renewcommand{\arraystretch}{1.0}
\vspace{-\baselineskip}

Модульный тест для функций eval\_node и parse\_command из модуля parser.py представлен в таблице~\ref{table:parser_test}.

\renewcommand{\arraystretch}{0.8}
\begin{xltabular}{\textwidth}{|X|X|X|X|}
	\caption{Модульное тестирование parser.py\label{table:parser_test}} \\
	\hline
	\centrow Описание теста &
	\centrow Тестируемая функция & 
	\centrow Входные данные & 
	\centrow Ожидаемый результат \\ 
	\hline 
	\endfirsthead
	
	\caption*{Продолжение таблицы \ref{table:parser_test}} \\
	\hline 
	\centrow Описание теста & 
	\centrow Тестируемая функция & 
	\centrow Входные данные & 
	\centrow Ожидаемый результат \\ 
	\hline 
	\endhead
	
	Проверка обработки констант разных типов & eval\_node & Константы разных типов & Значения соответствующих констант \\ \hline
	Проверка доступа к переменным в контексте & eval\_node & Имя переменной и контекст (словарь) & Значение переменной в заданном контексте \\ \hline
	Проверка бинарных операций & eval\_node & Выражения с математическими операторами (+, -, *, /, //, \%, **) & Корректные результаты арифметических операций \\ \hline
	Проверка операторов сравнения & eval\_node & Выражения с операторами сравнения (>, >=, <, <=, ==, !=) & Корректные результаты сравнений \\ \hline
	Проверка логических операторов & eval\_node & and и or с разными комбинациями значений & Корректные результаты логических операций \\ \hline
	Проверка унарных операторов & eval\_node & Унарные операторы (унарный минус, not), применённые к различным значениям & Корректные результаты применения унарных операторов \\ \hline
	Проверка обработки пустой команды & parse\_command & Пустая строка & Ошибка \\ \hline
	Проверка создания таблицы & parse\_command & 1. Корректная команда\newline2. Ошибки синтаксиса & 1. Таблица создана\newline2. Ошибка\\ \hline
	Проверка создания БД & parse\_command & Команда создания БД & Создан файл БД, возвращен объект Database \\ \hline
	Проверка вставки данных & parse\_command & 1. Корректная команда\newline2. Ошибки синтаксиса & 1. Данные добавлены\newline2. Ошибка\\ \hline
	Проверка выборки данных & parse\_command & 1. Корректная команда\newline2. Ошибки синтаксиса & 1. Выбраны корректные данные\newline2. Ошибка\\ \hline
	Проверка обновления данных & parse\_command & 1. Корректная команда\newline2. Ошибки синтаксиса & 1. Данные обновлены\newline2. Ошибка\\ \hline
	Проверка удаления данных & parse\_command & 1. Корректная команда\newline2. Ошибки синтаксиса & 1. Данные удалены\newline2. Ошибка\\ \hline
	Проверка удаления таблицы & parse\_command & 1. Корректная команда\newline2. Ошибки синтаксиса & 1. Таблица удалена, при попытке обращения к ней выдаётся ошибка\newline2. Ошибка\\ \hline
\end{xltabular}
\renewcommand{\arraystretch}{1.0}
\vspace{-\baselineskip}

\subsection{Системное тестирование разработанной программной системы}

Для проведения системного тестирования был использован файл базы данных, содержащий таблицу с сотней записей.

На рисунке~\ref{fig:dbms_window} представлено главное окно СУБД.
\begin{figure}[H]
	\centering
	\includegraphics[width=0.9\linewidth]{"images/окно субд"}
	\caption{Главное окно программы}
	\label{fig:dbms_window}
\end{figure}

На рисунке~\ref{fig:new_db} показана информация о файле новой базы данных, созданной после ввода команды create database new\_database.
\begin{figure}[H]
	\centering
	\includegraphics[width=0.9\linewidth]{"images/новая бд"}
	\caption{Файл новой базы данных}
	\label{fig:new_db}
\end{figure}

На рисунке~\ref{fig:select} представлено окно выбора файла после нажатия на пункт меню <<Выбрать БД>>.
\begin{figure}[H]
	\centering
	\includegraphics[width=0.9\linewidth]{"images/выбор файла"}
	\caption{Окно выбора файла}
	\label{fig:select}
\end{figure}

На рисунке~\ref{fig:select_sql} представлен результат выборки данных из таблицы.
\begin{figure}[H]
	\centering
	\includegraphics[width=0.9\linewidth]{"images/выборка"}
	\caption{Результат выборки}
	\label{fig:select_sql}
\end{figure}

На рисунке~\ref{fig:insert_sql} представлен результат вставки данных в таблицу.
\begin{figure}[H]
	\centering
	\includegraphics[width=0.9\linewidth]{"images/вставка"}
	\caption{Результат вставки}
	\label{fig:insert_sql}
\end{figure}

На рисунке~\ref{fig:update_sql} представлен результат обновления данных в таблице.
\begin{figure}[H]
	\centering
	\includegraphics[width=0.9\linewidth]{"images/обновление"}
	\caption{Результат обновления}
	\label{fig:update_sql}
\end{figure}

На рисунке~\ref{fig:delete_sql} представлен результат удаления данных из таблицы.
\begin{figure}[H]
	\centering
	\includegraphics[width=0.9\linewidth]{"images/удаление"}
	\caption{Результат удаления}
	\label{fig:delete_sql}
\end{figure}

На рисунке~\ref{fig:error} представлено сообщение об ошибке при неправильно введённой команде выборки.
\begin{figure}[H]
	\centering
	\includegraphics[width=0.7\linewidth]{"images/ошибка"}
	\caption{Ошибка в команде выборки}
	\label{fig:error}
\end{figure}

На рисунке~\ref{fig:error_insert} представлено сообщение об ошибке при неправильно введённой команде вставки.
\begin{figure}[H]
	\centering
	\includegraphics[width=0.7\linewidth]{"images/ошибка вставки"}
	\caption{Ошибка в команде вставки}
	\label{fig:error_insert}
\end{figure}

На рисунке~\ref{fig:error_update} представлено сообщение об ошибке при неправильно введённой команде обновления.
\begin{figure}[H]
	\centering
	\includegraphics[width=0.7\linewidth]{"images/ошибка обновления"}
	\caption{Ошибка в команде обновления}
	\label{fig:error_update}
\end{figure}

На рисунке~\ref{fig:error_delete} представлено сообщение об ошибке при неправильно введённой команде удаления.
\begin{figure}[H]
	\centering
	\includegraphics[width=0.7\linewidth]{"images/ошибка удаления"}
	\caption{Ошибка в команде удаления}
	\label{fig:error_delete}
\end{figure}

На рисунке~\ref{fig:error_table} представлено сообщение об ошибке при попытке обращения к несуществующей таблице.
\begin{figure}[H]
	\centering
	\includegraphics[width=0.7\linewidth]{"images/ошибка таблица"}
	\caption{Обращение к несуществующей таблице}
	\label{fig:error_table}
\end{figure}



\subsection{Сборка программной системы}

Программные компоненты представляют собой файлы исходных кодов программной системы.

Для сборки и компиляции программной системы использовалась библиотека Pyinstaller, позволяющая упаковать все необходимые файлы в один исполняемый файл формата .exe. Данный файл может быть запущен без предварительной установки.

Интерпретация исходных кодов на языке Python выполняется встроенным в исполняемый файл интерпретатором языка и не требует отдельной установки интерпретатора и библиотек на целевую систему.

Все программные компоненты собраны в один исполняемый файл, готовый к запуску в среде Windows.