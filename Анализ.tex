\section{Анализ предметной области}
\subsection{Понятие и принципы работы баз данных}

База данных (БД) представляет собой структурированную совокупность взаимосвязанных данных, организованную таким образом, чтобы обеспечить их эффективное хранение, модификацию и извлечение при необходимости. Она служит основой для информационных систем различных сфер деятельности -- от бухгалтерии и логистики до здравоохранения и оборонной промышленности. Современные БД являются неотъемлемой частью цифровой инфраструктуры и используются в банках, интернет-магазинах, мобильных приложениях, государственных учреждениях и множестве других направлений.

С технической точки зрения, база данных -- это набор логически связанных данных, сопровождаемых программными средствами для их обработки. Они хранятся в виде записей, организованных в таблицы, индексы, схемы и представления. Такие структуры облегчают доступ и манипуляцию данными.	Обычно используется табличная модель (реляционная), где строки представляют записи, а столбцы — поля. Однако современные подходы также включают графовые, документо-ориентированные и объектные модели данных.
		
Все операции в базе данных могут быть объединены в транзакции -- логически завершённые единицы работы. Они обеспечивают целостность данных даже при сбоях. ACID-свойства транзакций гарантируют атомарность, согласованность, изолированность и долговечность выполнения операций. В многопользовательской среде возможны одновременные запросы к одной и той же информации. СУБД обеспечивает согласованность данных при параллельной работе нескольких пользователей путём блокировок, сериализации транзакций и версионного контроля.

ACID -- это аббревиатура, которая описывает четыре ключевых свойства транзакций в реляционных базах данных: Atomicity (атомарность), Consistency (согласованность), Isolation (изолированность), Durability (долговечность). Эти свойства гарантируют надежность обработки данных даже в случае сбоев, ошибок или параллельной работы пользователей.

Атомарность гарантирует, что транзакция является неделимой единицей выполнения: либо все изменения, входящие в транзакцию, применяются к базе данных, либо ни одно из них не применяется.

Согласованность гарантирует, что выполнение транзакции переводит базу данных из одного корректного состояния в другое, не нарушающее определённых целостностных ограничений.

Изолированность означает, что параллельно выполняющиеся транзакции не должны влиять друг на друга, и каждая из них должна выполняться так, как если бы она была единственной в системе.

Долговечность означает, что после фиксации транзакции её изменения становятся постоянными и не могут быть утеряны даже в случае сбоя системы.

Для манипуляции данными в БД используются особые языки.	Наиболее широко используется язык SQL (Structured Query Language), который позволяет описывать, изменять и извлекать информацию из базы данных.	
		
Для ускорения поиска и сортировки в структуре данных используются индексы. Они строятся по ключевым полям и значительно уменьшают объём операций при выборке.

Базы данных поддерживают средства резервного копирования, восстановления после сбоев, а также системы шифрования и разграничения прав доступа.

Современные тенденции развития ИТ вносят свои коррективы в классические принципы баз данных. Всё чаще применяются распределённые модели хранения, ориентированные на горизонтальное масштабирование, а также механизмы автоматической балансировки нагрузки и самовосстановления.

Существуют также принципы BASE, которые противопоставляются ACID и применяются в системах, ориентированных на высокую доступность и масштабируемость. BASE означает:
\begin{itemize}
	\item Basically Available -- система доступна даже при частичных сбоях;
	\item Soft state -- состояние системы может изменяться со временем;
	\item Eventually consistent -- система достигает согласованного состояния позже, а не мгновенно.
 \end{itemize}
 
Принципы BASE широко используются в NoSQL-хранилищах, особенно в крупных распределённых веб-приложениях.
 
\subsection{Системы управления базами данных (СУБД)}

Системы управления базами данных (СУБД) представляют собой специализированное программное обеспечение, предназначенное для создания, ведения, поддержки и взаимодействия с базами данных. Они выполняют роль посредника между конечным пользователем и базой данных, управляя всей информацией, обеспечивая её безопасность, целостность и доступность.

СУБД можно считать ядром большинства современных информационных систем. Практически каждое приложение, использующее хранилище данных — от банковской системы до мобильного сервиса доставки — так или иначе использует СУБД для структурированной работы с данными.

\subsubsection{Основные компоненты СУБД}

В основные компоненты СУБД входят её ядро, языки определения, манипулирования и управления данными и механизмы безопасности.

Ядро СУБД включает в себя компоненты для управления транзакциями, буферным кэшем, взаимодействием с файловой системой, выполнением запросов и их оптимизацией.

Язык определения данных (DDL) позволяет описывать структуру базы данных: таблицы, поля, индексы, ограничения.
	
Язык манипулирования данными (DML) используется для выполнения операций вставки, удаления, изменения и выборки данных.
	
Язык управления данными (DCL) обеспечивает управление доступом к объектам базы данных.
	
В механизмы безопасности входят системы аутентификации, шифрования, ведения журналов событий и разграничения прав пользователей.

\subsubsection{Классификация СУБД}

\paragraph{По модели данных}

По модели данных можно выделить реляционные, объектно-ориентированные, иерархические, сетевые и NoSQL СУБД.

Реляционные СУБД основаны на табличной модели, где данные организованы в виде связанных таблиц. Обеспечивают высокую степень согласованности данных и поддерживают язык SQL. Широко применяются в корпоративных системах. Примерами таких СУБД являются PostgreSQL, Oracle, MS SQL Server.

Объектно-ориентированные СУБД хранят данные в виде объектов, аналогично объектам в объектно-ориентированном программировании. Поддерживают наследование, инкапсуляцию и полиморфизм, что упрощает работу с комплексными структурами. В качестве примера таких СУБД можно привести db4o, ObjectDB.

Иерархические СУБД используют древовидную структуру для организации данных. Каждая запись имеет одну родительскую и множество дочерних записей. Подход эффективен для строго структурированных данных. В качестве можно привести IBM IMS.

Сетевые СУБД позволяют каждой записи иметь множество связей как с родителями, так и с дочерними элементами, что делает их более гибкими по сравнению с иерархическими СУБД. Примером такой СУБД является Integrated Data Store (IDS).

NoSQL-СУБД представляют собой альтернативу реляционным СУБД и включают несколько типов в себя документо-ориентированные (например, MongoDB), графовые (Neo4j), wide-column (Cassandra) СУБД и СУБД вида ключ-значение (Redis).

\paragraph{По способу размещения}

По способу размещения выделяют локальные, клиент-серверные и облачные СУБД.

Локальные СУБД устанавливаются на персональные компьютеры и используются в рамках одного пользователя или небольших рабочих групп. Подходят для настольных приложений и прототипирования. Пример: SQLite.

Клиент-серверные СУБД работают по модели, в которой сервер отвечает за хранение и обработку данных, а клиенты взаимодействуют с ним через сеть. Обеспечивают многопользовательский доступ и масштабируемость. Примерами являются PostgreSQL, MySQL, MS SQL Server.

Облачные СУБД (DBaaS – Database as a Service) предоставляются как услуга через облачные платформы. Пользователи не занимаются настройкой и обслуживанием серверов – это берет на себя провайдер. Обеспечивают гибкость, высокую доступность и автоматическое масштабирование. Примеры таких СУБД: Amazon RDS, Azure SQL Database, Yandex Managed PostgreSQL.

\paragraph{По способу хранения данных}

По способу хранения СУБД делят на in-memory и disk-based. 

In-memory СУБД хранят данные в оперативной памяти, что обеспечивает минимальные задержки при доступе. Подходят для высоконагруженных систем, где важна скорость обработки. Однако данные теряются при перезагрузке, если не реализовано сохранение на диск. Примеры: Redis, Tarantool.

Disk-based СУБД хранят данные на жёстких или твердотельных накопителях, что делает их более надёжными в долгосрочной перспективе. Поддерживают большие объемы данных и широко применяются в промышленной эксплуатации. Примеры: PostgreSQL, MySQL, Cassandra.

\subsubsection{Наиболее известные СУБД}

В качестве наиболее известных и используемых СУБД можно выделить следующие:

\begin{enumerate}
	\item PostgreSQL -- мощная объектно-реляционная СУБД с открытым исходным кодом, активно используемая в научных и коммерческих проектах.
	\item Oracle Database -- коммерческая СУБД с расширенными функциями безопасности, высокой надёжностью и поддержкой больших данных. Используется крупнейшими банками мира, поскольку обеспечивает высокую степень безопасности, аудит и соответствие нормативным требованиям.
	\item MySQL/MariaDB -- легковесные, но функциональные СУБД, популярные среди веб-разработчиков.
	\item MongoDB -- документо-ориентированная NoSQL СУБД, активно используется в проектах, связанных с большими данными и быстрым прототипированием. Используется компаниями вроде Facebook для хранения огромного объема пользовательских данных.
\end{enumerate}

\subsubsection{Выбор СУБД}

При выборе системы управления базами данных важно учитывать объём и характер данных (структурированные или неструктурированные), требования к надёжности и отказоустойчивости, предполагаемую нагрузку и количество пользователей, а также совместимость с существующей ИТ-инфраструктурой и потенциальные возможности масштабирования. 

Всё чаще организации отдают предпочтение гибридным решениям, сочетающим реляционные и NoSQL-подходы для разных компонентов информационной системы, что позволяет гибко адаптироваться к разнообразным бизнес-задачам.

\subsection{История развития систем хранения и управления данными}

Развитие систем хранения и управления данными неразрывно связано с эволюцией вычислительной техники и информационных технологий. С момента появления первых компьютеров человечество стремилось упорядочить, хранить и обрабатывать данные всё более эффективно.

\subsubsection{Ручная обработка данных (до 1950-х годов)}

До появления компьютеров данные хранились в бумажных архивах, бухгалтерских книгах и картотеках. Обработка информации осуществлялась вручную или с использованием механических счётных машин. Этот этап отличался высокой трудоёмкостью и низкой скоростью обработки информации, что сдерживало развитие крупных предприятий и систем управления.

\subsubsection{Появление электронных ЭВМ и файловых систем (1950--1960-е годы)}

С появлением первых электронных вычислительных машин (например, ENIAC, UNIVAC) возникла потребность в автоматизации хранения данных. В этот период данные стали сохраняться на магнитных лентах, позже -- на дисках, в виде файлов. Обработка осуществлялась с помощью процедурных языков (COBOL, FORTRAN), а доступ к данным — посредством файловых систем.

Проблемы этого этапа:
\begin{itemize}
	\item отсутствие централизованного управления данными;
	\item высокая избыточность информации;
	\item трудности при обновлении и сопровождении программ.
\end{itemize}

\subsubsection{Иерархические и сетевые СУБД (1960--1970-е годы)}

Понимая недостатки работы с файлами, разработчики начали создавать первые системы управления базами данных. В 1960-е годы появилась иерархическая модель данных, использующая древовидную структуру. Пример -- IBM IMS (Information Management System), применяемая в аэрокосмической отрасли.

Параллельно развивалась сетевая модель, в которой связи между записями описывались множественными отношениями. Пример -- Integrated Data Store (IDS), разработанный в General Electric. Сетевые и иерархические модели требовали от программистов точного знания структуры базы, что усложняло разработку и сопровождение.

\subsubsection{Реляционная модель данных (1970--1980-е годы)}

Ключевой революцией в области баз данных стало предложение Эдгара Ф. Кодда в 1970 году реляционной модели данных. Она основывалась на теории множеств и математической логике, что обеспечивало более гибкий и формальный подход к организации информации.

Основные идеи реляционной модели:
\begin{itemize}
	\item данные хранятся в виде таблиц (отношений);
	\item каждая таблица имеет уникальный ключ;
	\item связи между таблицами выражаются через внешние ключи;
	\item манипулирование данными осуществляется с помощью SQL.
\end{itemize}

В 1979 году была выпущена первая коммерческая реляционная СУБД -- Oracle. Позже появились IBM DB2, Microsoft SQL Server, Informix, Sybase и другие. Реляционные базы данных стали доминировать в корпоративной среде, обеспечивая высокую степень формализации, целостности и устойчивости к ошибкам.

\subsubsection{Расширение функциональности СУБД (1990--2000-е годы)}

На этом этапе реляционные СУБД стали развиваться по нескольким направлениям:
\begin{itemize}
	\item поддержка объектов (объектно-реляционные БД);
	\item расширение SQL (хранимые процедуры, триггеры);
	\item развитие репликации, кластеризации и масштабирования;
	\item упрощение администрирования и повышение безопасности.
\end{itemize}

Появились концепции «информационных хранилищ» (data warehouse), ориентированных на аналитическую обработку больших объёмов данных. Были внедрены OLAP-кубы, позволяющие быстро анализировать многомерные данные.

\subsubsection{Появление NoSQL и Big Data (2000--2010-е годы)}

С началом XXI века и развитием интернета, социальных сетей и мобильных устройств резко возрос объём, разнообразие и скорость появления данных (т.н. три V -- Volume, Variety, Velocity). Классические СУБД оказались неэффективными для масштабируемой обработки таких данных, и возник спрос на альтернативные решения.

Так появились:
\begin{itemize}
	\item документо-ориентированные БД (MongoDB, CouchDB);
	\item key-value хранилища (Redis, Riak);
	\item колонковые БД (Cassandra, HBase);
	\item графовые БД (Neo4j);
	\item поисковые движки (Elasticsearch).
\end{itemize}

NoSQL-СУБД отказались от строгой схемы и обеспечили высокую масштабируемость, что особенно востребовано в распределённых системах и облачных сервисах.

\subsubsection{Современные тенденции (2010-е -- по настоящее время)}

Современный этап развития систем управления базами данных характеризуется стремлением объединить преимущества реляционных и нереляционных подходов. Появление NewSQL-решений, таких как Google Spanner и CockroachDB, направлено на сохранение строгих ACID-гарантий при обеспечении высокой масштабируемости.

Широкое распространение получил подход DBaaS (Database as a Service), при котором базы данных предоставляются как облачные сервисы — примерами служат Amazon Aurora и Yandex Managed PostgreSQL.

Современные СУБД поддерживают разнообразные форматы хранения, включая JSON, XML и геоданные, что делает их гибкими и адаптивными к различным типам информации. 

Всё активнее внедряются технологии искусственного интеллекта и машинного обучения, позволяющие автоматически оптимизировать запросы и проводить продвинутую аналитику больших данных. Кроме того, развитие контейнеризации и микросервисной архитектуры привело к тому, что базы данных разворачиваются в средах вроде Kubernetes и управляются средствами инфраструктуры как кода.

История СУБД — это история постоянной эволюции от централизованных монолитных систем к гибким, масштабируемым, распределённым решениям, адаптированным к требованиям цифровой эпохи.

\subsection{Системы управления базами данных в России}

Российский рынок систем управления базами данных развивался под влиянием как внутренних научно-технических достижений, так и глобальных мировых трендов. В течение долгого времени российские организации активно использовали западные СУБД, такие как Oracle, Microsoft SQL Server, PostgreSQL. Однако в условиях нарастающего внимания к вопросам импортозамещения, информационной безопасности и технологического суверенитета в последние годы наблюдается активное развитие отечественных решений.

\subsubsection{Исторический контекст}

В СССР велась масштабная работа в области кибернетики и автоматизации. Уже в 1960--1970-е годы существовали отечественные разработки в области баз данных, такие как ИАС (информационно-алфавитные системы), а также специализированные базы данных, создаваемые для военных и научных целей. Однако доступ к западным наработкам был ограничен, а внутренние решения не получили широкого распространения за пределами оборонной и академической сферы.

После распада СССР, в 1990-х годах, в условиях рыночной экономики российские организации стали массово внедрять коммерческие зарубежные СУБД -- Oracle, Microsoft SQL Server, IBM DB2 и др. Это сопровождалось развитием IT-консалтинга, аутсорсинга и роста потребности в специалистах по базам данных.

\subsubsection{Современные отечественные СУБД}

На фоне необходимости импортозамещения государственные программы и крупные корпорации начали активно инвестировать в разработку и внедрение отечественных систем управления базами данных. Это способствовало появлению и развитию целого ряда российских СУБД, адаптированных под требования национальной инфраструктуры, информационной безопасности и специфики использования в госсекторе и промышленности.

Postgres Pro — одна из самых известных отечественных СУБД, разрабатываемая компанией Postgres Professional. Она основана на международной версии PostgreSQL, но включает уникальные доработки, такие как повышенная производительность, усиленные механизмы безопасности и сертификация ФСТЭК и ФСБ. Postgres Pro активно применяется в органах государственного управления, банковской сфере и промышленности.

Линтер — это СУБД реального времени, разработанная НПП «РЕЛЭКС». Её отличает высокий уровень безопасности, что делает её востребованной в автоматизированных системах управления технологическими процессами. Линтер сертифицирована для работы с государственной тайной и используется в критически важных инфраструктурах.

Ред База Данных (Red Database) — российская версия СУБД Firebird, созданная компанией РедСофт. Она сертифицирована Минцифры РФ и применяется в различных государственных проектах. Red Database сохраняет совместимость с Firebird, но адаптирована к требованиям российского законодательства и стандартов безопасности.

Tarantool — это высокопроизводительная in-memory СУБД, разработанная компанией Mail.ru Group (ныне VK). Она поддерживает язык Lua для написания логики на стороне сервера и хорошо масштабируется, что делает её подходящей для работы с высоконагруженными онлайн-сервисами, в том числе в сфере электронной коммерции и цифровых платформ.

Базис — отечественная СУБД, созданная в НИИСИ РАН. Она ориентирована на применение во встроенных системах и военной технике, где особенно важны отказоустойчивость, защищённость и надёжность. Базис используется в специализированных проектах, требующих высокой степени доверия к программному обеспечению.

\subsubsection{Проблемы и вызовы}

Несмотря на развитие отечественных СУБД, их широкому внедрению препятствует ряд серьёзных проблем. Одной из ключевых является ограниченная экосистема: по сравнению с такими зрелыми решениями, как PostgreSQL или Oracle, российские СУБД располагают менее развитым набором модулей, инструментов и интеграций.

Кроме того, существует дефицит квалифицированных специалистов, так как большинство кадров традиционно обучены работе с зарубежными системами.

Немаловажным фактором является и консерватизм организаций — переход на отечественные решения требует значительных затрат на миграцию, тестирование и переобучение персонала. 

В ответ на эти вызовы государство предпринимает меры стимулирования: введён перечень отечественного ПО, обязательного к использованию в госорганах, реализуются программы субсидирования миграции с иностранного ПО, а также активно развиваются национальные цифровые платформы — такие как «ГосТех», «Цифровой профиль гражданина» и «Единая система электронных документов», — которые опираются на использование российских СУБД.

\subsubsection{Тенденции развития}

В настоящее время имеются следующие тенденции развития систем хранения и управления данными в России:

\begin{enumerate}
	\item Активизация разработки NoSQL и NewSQL решений.
	\item Усиление связки СУБД с инструментами аналитики и BI.
	\item Повышение требований к безопасности: сертификация по ГОСТ, соответствие требованиям к защите персональных данных.
	\item Рост сотрудничества между государством и частным ИТ-сектором.	
\end{enumerate}

В России формируется собственная экосистема СУБД, способная обеспечить базовые и специализированные задачи управления данными без зависимости от зарубежного программного обеспечения. Это важно как в контексте кибербезопасности, так и с точки зрения технологического суверенитета.

\subsection{Динамика и перспективы развития систем хранения и управления данными}

Мир баз данных постоянно меняется, подстраиваясь под потребности бизнеса, технологий и общества. Сегодня наблюдается стремительное развитие как классических СУБД, так и новых гибридных архитектур, затрагивающих распределённые системы, облачные вычисления, машинное обучение и кибербезопасность.

\subsubsection{Рост объёмов данных и потребность в масштабируемости}

Согласно исследованиям IDC, к 2025 году объем глобальных данных может превысить 180 зеттабайт. Это требует:
\begin{itemize}
	\item масштабируемых хранилищ;
	\item распределённой архитектуры;
	\item отказоустойчивых кластеров;
	\item систем с горизонтальным масштабированием.
\end{itemize}

Системы наподобие Google Bigtable, Amazon DynamoDB, Apache Cassandra предлагают решения, обеспечивающие хранение петабайт информации с высокой доступностью и низкой задержкой.

\subsubsection{Эволюция архитектур}

Эволюция архитектур СУБД смещается от традиционных централизованных серверов к более гибким и масштабируемым подходам. 

Современные решения всё чаще используют микросервисную архитектуру, где каждая часть бизнес-логики имеет собственное изолированное хранилище.

Контейнеризация и оркестрация позволяют управлять базами данных как сервисами, упрощая развертывание и масштабирование. 

Развитие edge computing способствует обработке данных ближе к их источникам — например, в устройствах IoT. Кроме того, концепция data mesh продвигает децентрализацию ответственности за данные между командами и сервисами, усиливая гибкость и управляемость распределённых систем.

\subsubsection{Облачные базы данных и модели DBaaS}

Базы данных как сервис (DBaaS) позволяют компаниям использовать мощные СУБД без необходимости управлять инфраструктурой. Примеры: Amazon Aurora, Google Cloud SQL, Azure Cosmos DB, Yandex Managed PostgreSQL, VK Cloud, Tarantool Cloud.

Преимущества модели DBaaS:
\begin{enumerate}
	\item Администрирование как сервис. Провайдер инфраструктуры выполняет все задачи, связанные с развёртыванием, настройкой и администрированием кластеров баз данных. 
	\item Высокая безопасность. Провайдеры работают в защищённых средах с использованием дополнительных мер (межсетевых экранов, антивирусов и т. д.).
	\item Доступность. Облачная база данных может быть доступна в любой момент для незамедлительных изменений -- для этого потребуется только подключение к сети и компьютер.
	\item Масштабируемость. Облачная база данных допускает масштабирование ресурсов непосредственно во время работы. Эта характеристика важна для организаций в моменты пиковых/сезонных нагрузок, а также для активно развивающихся и растущих компаний. 
\end{enumerate}

\subsubsection{Искусственный интеллект и автоматизация управления}

Современные СУБД всё активнее интегрируют технологии искусственного интеллекта и машинного обучения для повышения эффективности и автономности управления. Такие системы способны предсказывать и оптимизировать запросы, автоматически настраивать конфигурации, выявлять аномалии и возможные вторжения, а также упрощать процессы миграции и резервного копирования. Яркими примерами являются Google BigQuery, использующая ИИ для автотюнинга запросов, и Oracle Autonomous Database, сводящая к минимуму необходимость ручного вмешательства.

\subsubsection{Безопасность и соответствие требованиям законодательства}

С ростом числа кибератак и усилением требований со стороны законодательства — таких как GDPR, 152-ФЗ и HIPAA — безопасность систем управления базами данных становится приоритетной задачей. В ответ на эти вызовы наблюдаются устойчивые тенденции, направленные на повышение защищённости данных. Одной из таких тенденций является стандартизация механизмов шифрования непосредственно на уровне СУБД, что позволяет защищать информацию как при хранении, так и при передаче. 

Всё шире внедряются ролевые модели управления доступом, дополняемые многофакторной аутентификацией для повышения надежности идентификации пользователей. Также усиливается контроль за действиями внутри СУБД — реализуется обязательное протоколирование всех операций (audit trail), что обеспечивает прозрачность и возможность последующего анализа действий пользователей. 

Отдельное внимание уделяется защите данных в облачных средах: используются технологии, обеспечивающие их конфиденциальность и соответствие нормативным требованиям при размещении в инфраструктуре сторонних провайдеров.

\subsubsection{Перспективные направления}

В настоящее время активно развиваются несколько перспективных направлений в области систем управления базами данных, отражающих технологические тренды и будущие потребности цифровой инфраструктуры.

Одним из таких направлений являются квантовые базы данных (Quantum databases) — ведутся исследования и разработка прототипов СУБД, использующих принципы квантовых вычислений. Такие решения потенциально способны значительно ускорить обработку и поиск данных за счёт параллельной обработки на квантовом уровне.

Другое перспективное направление — децентрализованное хранение данных на основе блокчейна (Blockchain-based storage). В таких моделях обеспечивается верификация каждой транзакции и защита от несанкционированного изменения данных, что особенно актуально для финансовых систем и распределённых реестров. Примером является BigchainDB, сочетающая свойства СУБД и технологии блокчейн.

Digital Twin Storage — концепция специализированного хранения данных для цифровых двойников объектов. Это необходимо для создания точных виртуальных копий физических систем и их синхронизации в реальном времени, что находит применение в промышленности, медицине и инженерии.

BaaS (Blockchain-as-a-Service) предлагает возможность хранения данных в блокчейн-структуре с доступом через API. Это упрощает интеграцию блокчейн-технологий в бизнес-приложения, позволяя использовать преимущества распределённого хранения без необходимости разворачивать собственную инфраструктуру.

Эти направления указывают на будущее развитие СУБД, ориентированное на безопасность, распределённость, и масштабируемость.

\subsubsection{Прогнозы аналитиков}

Аналитики отмечают устойчивые тенденции, определяющие будущее развития рынка СУБД. 

По прогнозу Gartner, к 2030 году до 80\% коммерческих баз данных будут функционировать в облачной среде, что отражает общий курс на отказ от локальной инфраструктуры в пользу гибкости и масштабируемости облаков. 

Forrester указывает на растущий интерес к архитектурам типа «data fabric» и «data lakehouse», которые интегрируют хранилища данных, потоки и аналитику в единую, адаптивную среду. 

Аналитики Центра информационных стратегий (ЦИС) прогнозируют, что доля отечественных СУБД в государственных проектах в России достигнет 60\% уже к 2027 году, что обусловлено политикой импортозамещения и поддержкой отечественного ИТ-сектора.

Развитие систем хранения и управления данными движется в сторону универсальности, автономности, безопасности и масштабируемости. Рынок требует от разработчиков СУБД постоянного обновления архитектурных подходов, поддержки новых форматов и интеграции с высокоуровневыми аналитическими инструментами. Это делает область баз данных не только критически важной для цифровой трансформации, но и одной из самых динамично развивающихся сфер IT-индустрии.