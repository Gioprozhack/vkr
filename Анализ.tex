\section{Анализ предметной области}
\subsection{Понятие и принципы работы баз данных}

База данных (БД) представляет собой структурированную совокупность взаимосвязанных данных, организованную таким образом, чтобы обеспечить их эффективное хранение, модификацию и извлечение при необходимости. Она служит основой для информационных систем различных сфер деятельности — от бухгалтерии и логистики до здравоохранения и оборонной промышленности. Современные БД являются неотъемлемой частью цифровой инфраструктуры и используются в банках, интернет-магазинах, мобильных приложениях, государственных учреждениях и множестве других направлений.

С технической точки зрения, база данных — это набор логически связанных данных, сопровождаемых программными средствами для их обработки. Они хранятся в виде записей, организованных в таблицы, индексы, схемы и представления. Такие структуры облегчают доступ и манипуляцию данными.

Принципы работы баз данных:
\begin{enumerate}
	\item Организация данных в логические структуры. Обычно используется табличная модель (реляционная), где строки представляют записи, а столбцы — поля. Однако современные подходы также включают графовые, документо-ориентированные и объектные модели данных.
	\item Поддержка транзакционности. Все операции в базе данных могут быть объединены в транзакции — логически завершённые единицы работы. Они обеспечивают целостность данных даже при сбоях. ACID-свойства транзакций гарантируют атомарность, согласованность, изолированность и долговечность выполнения операций.
	\item Механизмы управления параллелизмом. В многопользовательской среде возможны одновременные запросы к одной и той же информации. СУБД обеспечивает согласованность данных при параллельной работе нескольких пользователей путём блокировок, сериализации транзакций и версионного контроля.
	\item Поддержка языка манипулирования данными. Наиболее широко используется язык SQL (Structured Query Language), который позволяет описывать, изменять и извлекать информацию из базы данных.
	\item Оптимизация доступа и индексирование. Для ускорения поиска и сортировки данных используются индексы. Они строятся по ключевым полям и значительно уменьшают объём операций при выборке.
	\item Обеспечение надежности и безопасности хранения. Базы данных поддерживают средства резервного копирования, восстановления после сбоев, а также системы шифрования и разграничения прав доступа.
\end{enumerate}

ACID — это аббревиатура, которая описывает четыре ключевых свойства транзакций в реляционных базах данных: Atomicity (атомарность), Consistency (согласованность), Isolation (изолированность), Durability (долговечность). Эти свойства гарантируют надежность обработки данных даже в случае сбоев, ошибок или параллельной работы пользователей.
\begin{enumerate}
	\item Атомарность гарантирует, что транзакция является неделимой единицей выполнения: либо все изменения, входящие в транзакцию, применяются к базе данных, либо ни одно из них не применяется.
	\item Согласованность гарантирует, что выполнение транзакции переводит базу данных из одного корректного состояния в другое, не нарушающее определённых целостностных ограничений.
	\item Изолированность означает, что параллельно выполняющиеся транзакции не должны влиять друг на друга, и каждая из них должна выполняться так, как если бы она была единственной в системе.
	\item Долговечность означает, что после фиксации транзакции её изменения становятся постоянными и не могут быть утеряны даже в случае сбоя системы.
\end{enumerate}

Современные тенденции развития ИТ вносят свои коррективы в классические принципы баз данных. Всё чаще применяются распределённые модели хранения, ориентированные на горизонтальное масштабирование, а также механизмы автоматической балансировки нагрузки и самовосстановления.

Существуют также принципы BASE, которые противопоставляются ACID и применяются в системах, ориентированных на высокую доступность и масштабируемость. BASE означает:
\begin{itemize}
	\item Basically Available -- система доступна даже при частичных сбоях;
	\item Soft state — состояние системы может изменяться со временем;
	\item Eventually consistent — система достигает согласованного состояния позже, а не мгновенно.
 \end{itemize}
 
Принципы BASE широко используются в NoSQL-хранилищах, особенно в крупных распределённых веб-приложениях.
 
\subsection{Системы управления базами данных (СУБД)}

Системы управления базами данных (СУБД) представляют собой специализированное программное обеспечение, предназначенное для создания, ведения, поддержки и взаимодействия с базами данных. Они выполняют роль посредника между конечным пользователем и базой данных, управляя всей информацией, обеспечивая её безопасность, целостность и доступность.

СУБД можно считать ядром большинства современных информационных систем. Практически каждое приложение, использующее хранилище данных — от банковской системы до мобильного сервиса доставки — так или иначе использует СУБД для структурированной работы с данными.

\subsubsection{Основные компоненты СУБД}

\begin{enumerate}
	\item Ядро СУБД — включает компоненты для управления транзакциями, буферным кэшем, взаимодействием с файловой системой, выполнением запросов и их оптимизацией.
	\item Язык определения данных (DDL) — позволяет описывать структуру базы данных: таблицы, поля, индексы, ограничения.
	\item Язык манипулирования данными (DML) — используется для выполнения операций вставки, удаления, изменения и выборки данных.
	\item Язык управления данными (DCL) — обеспечивает управление доступом к объектам базы данных.
	\item Управление транзакциями — компонент, отвечающий за согласованное выполнение последовательностей операций.
	\item Механизмы безопасности — системы аутентификации, шифрования, ведения журналов событий и разграничения прав пользователей.
\end{enumerate}

\subsubsection{Классификация СУБД}

СУБД можно классифицировать по различным признакам:
\begin{enumerate}
	\item По модели данных:
	\begin{itemize}
		\item реляционные СУБД (RDBMS) — основаны на табличной модели (PostgreSQL, Oracle, MS SQL Server);
		\item объектно-ориентированные СУБД — хранят данные в виде объектов, как в ООП (db4o, ObjectDB);
		\item иерархические СУБД — данные организованы по иерархической модели (IBM IMS);
		\item сетевые СУБД — сложные связи между записями (Integrated Data Store);
		\item NoSQL-СУБД — документо-ориентированные, графовые, key-value и wide-column СУБД (MongoDB, Redis, Neo4j, Cassandra).
	\end{itemize}
	\item По способу размещения:
	\begin{itemize}
		\item локальные СУБД — устанавливаются на персональные компьютеры, часто используются в настольных приложения;
		\item клиент-серверные СУБД — сервер обрабатывает запросы от клиентов, обеспечивает масштабируемость и многопользовательский доступ;
		\item облачные СУБД (DBaaS) — базы данных предоставляются как услуга через облако, не требуют локальной установки и настройки (Amazon RDS, Azure SQL, Yandex Managed PostgreSQL).
	\end{itemize}
	\item По способу хранения данных:
	\begin{itemize}
		\item in-memory СУБД — хранят данные в оперативной памяти для высокой производительности (Tarantool, Redis);
		\item колонковые СУБД — хранят данные столбцами, а не строками, что делает их более оптимизированными для больших аналитических запросов (ClickHouse, Vertica, Tarantool).		
	\end{itemize}
\end{enumerate}

\subsubsection{Функциональные возможности современных СУБД}

\begin{enumerate}
	\item Хранимые процедуры и триггеры. Позволяют выполнять бизнес-логику на стороне сервера БД.
	\item Репликация и кластеризация. Используются для повышения отказоустойчивости и масштабируемости.
	\item Поддержка JSON, XML, геоданных. Расширяет возможности работы с неструктурированными и полуструктурированными данными.
	\item Параллельное выполнение запросов и шардирование. Используются в высоконагруженных системах.
\end{enumerate}

\subsubsection{Примеры СУБД}

\begin{enumerate}
	\item PostgreSQL — мощная объектно-реляционная СУБД с открытым исходным кодом, активно используемая в научных и коммерческих проектах.
	\item Oracle Database — коммерческая СУБД с расширенными функциями безопасности, высокой надёжностью и поддержкой больших данных.
	\item MySQL/MariaDB — легковесные, но функциональные СУБД, популярные среди веб-разработчиков.
	\item MongoDB — документо-ориентированная NoSQL СУБД, активно используется в проектах, связанных с большими данными и быстрым прототипированием.
\end{enumerate}

Примеры использования:
\begin{enumerate}
	\item Финансовый сектор: Oracle Database используется крупнейшими банками мира, поскольку обеспечивает высокую степень безопасности, аудит и соответствие нормативным требованиям.
	\item Электронная коммерция: PostgreSQL часто применяется в интернет-магазинах благодаря своей расширяемости и высокой производительности.
	\item Социальные сети и медиа: MongoDB и Cassandra используют компании вроде Facebook и Instagram для хранения огромного объема пользовательских данных.
\end{enumerate}

\subsubsection{Выбор СУБД}

При выборе СУБД необходимо учитывать:
\begin{itemize}
	\item объём и характер данных (структурированные или нет);
	\item требования к надёжности и отказоустойчивости;
	\item ожидаемую нагрузку и число пользователей;
	\item совместимость с существующей ИТ-инфраструктурой;
	\item возможности масштабирования.
\end{itemize}

Современные организации всё чаще делают выбор в пользу гибридных решений, совмещающих реляционные и NoSQL-подходы для разных компонентов информационной системы.

\subsection{История развития систем хранения и управления данными}

Развитие систем хранения и управления данными неразрывно связано с эволюцией вычислительной техники и информационных технологий. С момента появления первых компьютеров человечество стремилось упорядочить, хранить и обрабатывать данные всё более эффективно.

\subsubsection{Ручная обработка данных (до 1950-х годов)}

До появления компьютеров данные хранились в бумажных архивах, бухгалтерских книгах и картотеках. Обработка информации осуществлялась вручную или с использованием механических счётных машин. Этот этап отличался высокой трудоёмкостью и низкой скоростью обработки информации, что сдерживало развитие крупных предприятий и систем управления.

\subsubsection{Появление электронных ЭВМ и файловых систем (1950–1960-е годы)}

С появлением первых электронных вычислительных машин (например, ENIAC, UNIVAC) возникла потребность в автоматизации хранения данных. В этот период данные стали сохраняться на магнитных лентах, позже — на дисках, в виде файлов. Обработка осуществлялась с помощью процедурных языков (COBOL, FORTRAN), а доступ к данным — посредством файловых систем.

Проблемы этого этапа:
\begin{itemize}
	\item отсутствие централизованного управления данными;
	\item высокая избыточность информации;
	\item трудности при обновлении и сопровождении программ.
\end{itemize}

\subsubsection{Иерархические и сетевые СУБД (1960–1970-е годы)}

Понимая недостатки работы с файлами, разработчики начали создавать первые системы управления базами данных. В 1960-е годы появилась иерархическая модель данных, использующая древовидную структуру. Пример — IBM IMS (Information Management System), применяемая в аэрокосмической отрасли.

Параллельно развивалась сетевая модель, в которой связи между записями описывались множественными отношениями. Пример — Integrated Data Store (IDS), разработанный в General Electric. Сетевые и иерархические модели требовали от программистов точного знания структуры базы, что усложняло разработку и сопровождение.

\subsubsection{Реляционная модель данных (1970–1980-е годы)}

Ключевой революцией в области баз данных стало предложение Эдгара Ф. Кодда в 1970 году реляционной модели данных. Она основывалась на теории множеств и математической логике, что обеспечивало более гибкий и формальный подход к организации информации.

Основные идеи реляционной модели:
\begin{itemize}
	\item данные хранятся в виде таблиц (отношений);
	\item каждая таблица имеет уникальный ключ;
	\item связи между таблицами выражаются через внешние ключи;
	\item манипулирование данными осуществляется с помощью SQL.
\end{itemize}

В 1979 году была выпущена первая коммерческая реляционная СУБД -- Oracle. Позже появились IBM DB2, Microsoft SQL Server, Informix, Sybase и другие. Реляционные базы данных стали доминировать в корпоративной среде, обеспечивая высокую степень формализации, целостности и устойчивости к ошибкам.

\subsubsection{Расширение функциональности СУБД (1990–2000-е годы)}

На этом этапе реляционные СУБД стали развиваться по нескольким направлениям:
\begin{itemize}
	\item поддержка объектов (объектно-реляционные БД);
	\item расширение SQL (хранимые процедуры, триггеры);
	\item развитие репликации, кластеризации и масштабирования;
	\item упрощение администрирования и повышение безопасности.
\end{itemize}

Появились концепции «информационных хранилищ» (data warehouse), ориентированных на аналитическую обработку больших объёмов данных. Были внедрены OLAP-кубы, позволяющие быстро анализировать многомерные данные.

\subsubsection{Появление NoSQL и Big Data (2000–2010-е годы)}

С началом XXI века и развитием интернета, социальных сетей и мобильных устройств резко возрос объём, разнообразие и скорость появления данных (т.н. три V — Volume, Variety, Velocity). Классические СУБД оказались неэффективными для масштабируемой обработки таких данных, и возник спрос на альтернативные решения.

Так появились:
\begin{itemize}
	\item документо-ориентированные БД (MongoDB, CouchDB);
	\item key-value хранилища (Redis, Riak);
	\item колонковые БД (Cassandra, HBase);
	\item графовые БД (Neo4j);
	\item поисковые движки (Elasticsearch).
\end{itemize}

NoSQL-СУБД отказались от строгой схемы и обеспечили высокую масштабируемость, что особенно востребовано в распределённых системах и облачных сервисах.

\subsubsection{Современные тенденции (2010-е — по настоящее время)}

Современный этап характеризуется стремлением объединить достоинства реляционных и нереляционных решений:
\begin{itemize}
	\item NewSQL — попытка сохранить ACID-свойства при высокой масштабируемости (Google Spanner, CockroachDB);
	\item DBaaS — базы данных как сервис (Amazon Aurora, Yandex Managed PostgreSQL);
	\item многообразие форматов хранения — поддержка JSON, XML, геоданных;
	\item интеграция с ИИ и машинным обучением — автоматическая оптимизация запросов, аналитика больших данных;
	\item контейнеризация и микросервисная архитектура — базы данных разворачиваются в Kubernetes, управляются через инфраструктуру как код.
\end{itemize}

История СУБД — это история постоянной эволюции от централизованных монолитных систем к гибким, масштабируемым, распределённым решениям, адаптированным к требованиям цифровой эпохи.

\subsection{Системы управления базами данных в России}

Российский рынок систем управления базами данных развивался под влиянием как внутренних научно-технических достижений, так и глобальных мировых трендов. В течение долгого времени российские организации активно использовали западные СУБД, такие как Oracle, Microsoft SQL Server, PostgreSQL. Однако в условиях нарастающего внимания к вопросам импортозамещения, информационной безопасности и технологического суверенитета в последние годы наблюдается активное развитие отечественных решений.

\subsubsection{Исторический контекст}

В СССР велась масштабная работа в области кибернетики и автоматизации. Уже в 1960–1970-е годы существовали отечественные разработки в области баз данных, такие как ИАС (информационно-алфавитные системы), а также специализированные базы данных, создаваемые для военных и научных целей. Однако доступ к западным наработкам был ограничен, а внутренние решения не получили широкого распространения за пределами оборонной и академической сферы.

После распада СССР, в 1990-х годах, в условиях рыночной экономики российские организации стали массово внедрять коммерческие зарубежные СУБД — Oracle, Microsoft SQL Server, IBM DB2 и др. Это сопровождалось развитием IT-консалтинга, аутсорсинга и роста потребности в специалистах по базам данных.

\subsubsection{Современные отечественные СУБД}

На фоне необходимости импортозамещения, государственные программы и крупные корпорации начали вкладываться в разработку и внедрение отечественных СУБД. Среди наиболее известных российских решений:
\begin{enumerate}
	\item Postgres Pro:
	\begin{itemize}
		\item разрабатывается компанией Postgres Professional;
		\item основана на международной версии PostgreSQL, но содержит уникальные доработки: улучшенная производительность, усиленная безопасность, сертификация ФСТЭК и ФСБ;
		\item используется в органах госуправления, банковском секторе и промышленности.
	\end{itemize}
	\item Линтер:
	\begin{itemize}
		\item разработка НПП "<РЕЛЭКС">;
		\item СУБД реального времени с высоким уровнем безопасности, широко используется в автоматизированных системах управления технологическими процессами;
		\item сертифицирована для работы с гостайной.
	\end{itemize}
	\item Ред База Данных (Red Database):
	\begin{itemize}
		\item российская форк-версия Firebird, разработанная "<РедСофт">.
		\item имеет сертификацию Минцифры РФ и используется в государственных проектах.
	\end{itemize}
	\item Tarantool:
	\begin{itemize}
		\item разработка компании Mail.ru Group (VK);
		\item in-memory СУБД с поддержкой Lua-скриптов и высоким уровнем масштабируемости;
		\item используется в высоконагруженных онлайн-сервисах.
	\end{itemize}
	\item Базис:
	\begin{itemize}
		\item создана в НИИСИ РАН;
		\item подходит для встроенных систем и военных приложений, где важна отказоустойчивость и защищённость.
	\end{itemize}
\end{enumerate}

\subsubsection{Проблемы и вызовы}

Несмотря на наличие отечественных решений, существует ряд трудностей:
\begin{itemize}
	\item ограниченная экосистема. В отличие от PostgreSQL или Oracle, российские СУБД не имеют такого широкого набора модулей и интеграций;
	\item дефицит квалифицированных кадров. Большинство специалистов обучены работе с западными СУБД;
	\item консерватизм организаций. Переход на отечественные СУБД требует затрат на миграцию, тестирование и переобучение персонала.
\end{itemize}

Для стимулирования перехода на отечественные решения приняты следующие меры:
\begin{itemize}
	\item введение перечня отечественного ПО, обязательного к использованию в госорганах;
	\item программы субсидирования миграции с иностранного ПО на российское;
	\item развитие платформ "<ГосТех">, "<Цифровой профиль гражданина">, "<Единая система электронных документов">, использующих российские СУБД.
\end{itemize}

\subsubsection{Использование в различных отраслях}

\begin{enumerate}
	\item Госуправление: Postgres Pro и Ред БД применяются в системах электронного документооборота, реестрах и порталах госуслуг.
	\item Финансовый сектор: Линтер и Postgres Pro активно применяются в государственных банках и платёжных системах.
	\item Промышленность: Линтер и Базис обеспечивают устойчивую работу SCADA-систем и систем автоматизированного управления.
	\item Образование и наука: Tarantool и PostgreSQL используются в университетах и научных центрах.
\end{enumerate}

\subsubsection{Тенденции развития}

\begin{enumerate}
	\item Активизация разработки NoSQL и NewSQL решений.
	\item Усиление связки СУБД с инструментами аналитики и BI.
	\item Повышение требований к безопасности: сертификация по ГОСТ, соответствие требованиям к защите персональных данных.
	\item Рост сотрудничества между государством и частным ИТ-сектором.	
\end{enumerate}

Таким образом, в России формируется собственная экосистема СУБД, способная обеспечить базовые и специализированные задачи управления данными без зависимости от зарубежного программного обеспечения. Это важно как в контексте кибербезопасности, так и с точки зрения технологического суверенитета.

\subsection{Динамика и перспективы развития систем хранения и управления данными}

Мир баз данных постоянно меняется, подстраиваясь под потребности бизнеса, технологий и общества. Сегодня наблюдается стремительное развитие как классических СУБД, так и новых гибридных архитектур, затрагивающих распределённые системы, облачные вычисления, машинное обучение и кибербезопасность.

\subsubsection{Рост объёмов данных и потребность в масштабируемости}

Согласно исследованиям IDC, к 2025 году объем глобальных данных может превысить 180 зеттабайт. Это требует:
\begin{itemize}
	\item масштабируемых хранилищ;
	\item распределённой архитектуры;
	\item отказоустойчивых кластеров;
	\item систем с горизонтальным масштабированием.
\end{itemize}

Системы наподобие Google Bigtable, Amazon DynamoDB, Apache Cassandra предлагают решения, обеспечивающие хранение петабайт информации с высокой доступностью и низкой задержкой.

\subsubsection{Эволюция архитектур}

Традиционные СУБД строились на централизованных серверах. Современные подходы делают упор на:
\begin{itemize}
	\item микросервисы — каждая бизнес-логика имеет своё изолированное хранилище;
	\item контейнеризация и оркестрация — базы данных управляются как сервисы;
	\item edge computing — обработка данных ближе к источнику (датчики, IoT);
	\item data mesh — децентрализация владения и ответственности за данные.
\end{itemize}

\subsubsection{Облачные базы данных и модели DBaaS}

Базы данных как сервис (DBaaS) позволяют компаниям использовать мощные СУБД без необходимости управлять инфраструктурой. Примеры: Amazon Aurora, Google Cloud SQL, Azure Cosmos DB, Yandex Managed PostgreSQL, VK Cloud, Tarantool Cloud.

Преимущества модели DBaaS:
\begin{enumerate}
	\item Администрирование как сервис. Провайдер инфраструктуры выполняет все задачи, связанные с развёртыванием, настройкой и администрированием кластеров баз данных. 
	\item Высокая безопасность. Провайдеры работают в защищённых средах с использованием дополнительных мер (межсетевых экранов, антивирусов и т. д.).
	\item Доступность. Облачная база данных может быть доступна в любой момент для незамедлительных изменений — для этого потребуется только подключение к сети и компьютер.
	\item Масштабируемость. Облачная база данных допускает масштабирование ресурсов непосредственно во время работы. Эта характеристика важна для организаций в моменты пиковых/сезонных нагрузок, а также для активно развивающихся и растущих компаний. 
\end{enumerate}

\subsubsection{Конвергенция SQL и NoSQL (Multi-model БД)}

Современные приложения требуют гибкости. Появляется множество гибридных решений:
\begin{itemize}
	\item PostgreSQL с поддержкой JSON, hstore и геоданных;
	\item ArangoDB — хранение графов, документов и ключ-значений;
	\item OrientDB и MarkLogic — поддержка различных моделей в одной СУБД.
\end{itemize}

Такой подход позволяет использовать одну платформу для разных типов данных: транзакционных, аналитических, иерархических.

\subsubsection{Искусственный интеллект и автоматизация управления}

Новые СУБД активно внедряют машинное обучение для:
\begin{itemize}
	\item предсказания и оптимизации запросов;
	\item самостоятельной настройки конфигураций;
	\item обнаружения аномалий и вторжений;
	\item автоматизации миграций и резервного копирования.
\end{itemize}

Например, Google BigQuery использует искусственный интеллект для автотюнинга запросов, а Oracle Autonomous Database минимизирует необходимость ручной настройки.

\subsubsection{Безопасность и соответствие требованиям законодательства}

С ростом числа кибератак и ужесточением регулирования (GDPR, 152-ФЗ, HIPAA) наблюдается тенденция:
\begin{itemize}
	\item стандартизация шифрования данных на уровне СУБД;
	\item внедрение ролевых и многофакторных моделей доступа;
	\item протоколирование всех операций (audit trail);
	\item обеспечение конфиденциальности данных в облаке.
\end{itemize}

\subsubsection{Развитие аналитических и OLAP-систем}

Интерес к аналитике остаётся высоким. Активно развиваются:
\begin{itemize}
	\item колонковые базы данных (ClickHouse, Greenplum);
	\item интеграция СУБД с BI-платформами (Tableau, Power BI);
	\item применение in-memory баз (SAP HANA, Redis) для ускоренной аналитики.
\end{itemize}

\subsubsection{Перспективные направления}

\begin{itemize}
	\item Quantum databases — исследуются прототипы баз данных, использующих квантовые вычисления;
	\item Blockchain-based storage — децентрализованные модели хранения с верификацией транзакций (BigchainDB);
	\item Digital Twin Storage — хранение данных для цифровых двойников объектов;
	\item BaaS (Blockchain-as-a-Service) — хранение данных в цепочках блоков с API-доступом.	
\end{itemize}

\subsubsection{Прогнозы аналитиков}

\begin{itemize}
	\item Gartner предсказывает, что к 2030 году до 80\% коммерческих баз данных будут работать в облаке;
	\item Forrester отмечает увеличение интереса к «data fabric» и «data lakehouse» — архитектурам, объединяющим хранилища, потоки и аналитику;
	\item аналитики ЦИС (Центра информационных стратегий) прогнозируют рост доли отечественных СУБД в России до 60\% в государственных проектах к 2027 году.	
\end{itemize}

Развитие систем хранения и управления данными движется в сторону универсальности, автономности, безопасности и масштабируемости. Рынок требует от разработчиков СУБД постоянного обновления архитектурных подходов, поддержки новых форматов и интеграции с высокоуровневыми аналитическими инструментами. Это делает область баз данных не только критически важной для цифровой трансформации, но и одной из самых динамично развивающихся сфер IT-индустрии.